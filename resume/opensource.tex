%-------------------------------------------------------------------------------
%	SECTION TITLE
%-------------------------------------------------------------------------------
\cvsection{Projects}

\vspace{-2mm}

%-------------------------------------------------------------------------------
%	CONTENT
%-------------------------------------------------------------------------------
\begin{cventries}

%---------------------------------------------------------
  \cventry
    {} % Affiliation/role
    {Open-Source Projects} % Organization/group
    {} % Location
    {} % Date(s)
    {
      \vspace{-2mm}
      \begin{cvitems} % Description(s) of experience/contributions/knowledge
        \item {\textbf{PcapPlusPlus}: PcapPlusPlus is a multiplatform C++ library for capturing, parsing and crafting of network packets.
          Mujin's NetProbe project relies on this project for packets handling. While developing NetProbe, I found many places in PcapPlusPlus can be improved,
          so I keep contributing to this project, providing new features and bug fixes. I was granted as a \textbf{maintainer} of the project.
          Project link: \url{https://github.com/seladb/PcapPlusPlus/}.
        }
        \item {\textbf{modmesh}: modmesh is a toolkit for solving partial differential equations. It is written in C++ and pybind11 for high-Performance
        numerical computing. This project is contributed by many Taiwanese software engineers in semiconductor industry for fun. We aim to develop the 
        fastest and the most efficient numerical toolkit and we like to test new ideas on it. Project link: \url{https://github.com/solvcon/modmesh}.
        }
        \item {\textbf{node-color-log}: Owned a lightweight JavaScript logger with \textbf{46k/month} downloads,
          which provides an alternative solution to existing mature counterparts which usually have a huge codebase. Project link: \url{https://github.com/tigercosmos/node-color-log}.
        }
      \end{cvitems}
    }
  \vspace{-2mm}

\end{cventries}
