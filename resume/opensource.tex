%-------------------------------------------------------------------------------
%	SECTION TITLE
%-------------------------------------------------------------------------------
\cvsection{Open-Source Projects}

\vspace{-2mm}

%-------------------------------------------------------------------------------
%	CONTENT
%-------------------------------------------------------------------------------
\begin{cventries}

%---------------------------------------------------------
  \cventry
    {} % Affiliation/role
    {PcapPlusPlus} % Organization/group
    {} % Location
    {} % Date(s)
    {
      \vspace{-3mm}
      \begin{cvitems} % Description(s) of experience/contributions/knowledge
  \item {PcapPlusPlus is a multi-platform \textbf{C++ library} for capturing, parsing, and crafting \textbf{network packets}, with \textbf{3k stars} on GitHub.}
  \item {\textbf{One of five core maintainers} of PcapPlusPlus since 2024.}
  \item {Major contributions include fixing performance issues, establishing a more efficient CI/CD pipeline, adding new features, and improving documentation.}
        \item {Project link: \url{https://github.com/seladb/PcapPlusPlus/}.}
      \end{cvitems}
    }
    \vspace{2mm}

  \cventry
    {} % Affiliation/role
    {modmesh} % Organization/group
    {} % Location
    {} % Date(s)
    {
      \vspace{-3mm}
      \begin{cvitems} % Description(s) of experience/contributions/knowledge
  \item {modmesh is a toolkit for solving partial differential equations. It is written in \textbf{C++} and \textbf{pybind11} for high-performance numerical computing.}
  \item {Major contributions include a \textbf{low-level array implementation} similar to \textbf{NumPy}, leveraging C++ templates and Python bindings.}
  \item {Gave a talk at \textbf{PyCon JP 2024}: "Crafting Your Own NumPy: Do More in C++ and Make It Python".}
        \item {Project link: \url{https://github.com/solvcon/modmesh}.}
      \end{cvitems}
    }
    \vspace{2mm}

  % \cventry
  %   {} % Affiliation/role
  %   {node-color-log} % Organization/group
  %   {} % Location
  %   {} % Date(s)
  %   {
  %     \begin{cvitems} % Description(s) of experience/contributions/knowledge
  %       \item {
  %         I developed and maintained a lightweight \textbf{JavaScript} logger with \textbf{220k/month} downloads,
  %         which provides an alternative solution to existing mature counterparts which usually have a huge codebase.
  %       }
  %       \item {Project link: \url{https://github.com/tigercosmos/node-color-log}.}
  %     \end{cvitems}
  %   }

\end{cventries}
